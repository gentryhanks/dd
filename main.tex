%
% Hello! Here's how this works:
%
% You edit the source code here on the left, and the preview on the
% right shows you the result within a few seconds.
%
% Bookmark this page and share the URL with your co-authors. They can
% edit at the same time!
%
% You can upload figures, bibliographies, custom classes and
% styles using the files menu.
%
% If you're new to LaTeX, the wikibook at
% http://en.wikibooks.org/wiki/LaTeX
% is a great place to start, and there are some examples in this
% document, too.
%
% Enjoy!
%
\documentclass[12pt]{article}

\usepackage[english]{babel}
\usepackage[utf8x]{inputenc}
\usepackage{amsmath}
\usepackage{graphicx}
\usepackage{setspace} 
\usepackage{natbib}

\title{Household Medical Waste}
\author{gENTRY hANKS}

\begin{document}
\maketitle

\newpage
\begin{abstract}
\doublespacing 
My research examines flows of biomedical waste from domestic spaces (or on the go) to waste water treatment facilities, storm drain outlets, and landfills, in particular, biomedical waste generated by an ever increasing number of people with diabetes. According to the IDF (2012), diabetes is on the increase in every country. This increase necessarily means a growth in the production, consumption and thus the disposal of personal biomedical waste. Diabetes is a chronic illness that requires persistent management at home and on the go, which generates copious amounts of waste. These circumstances create a unique problem with tensions between product design (including the intended mode of disposal), personal practice and agency, government policy, dynamic technologies, social attitudes and waste management practices. This research seeks to understand the retention, disposal and ridding of medical waste in everyday life. The burden of waste currently falls on individuals that generate personal biomedical waste as well as waste facilities.  This, as well as social stigma, leads many individuals to simply throw away or flush much of the waste.  In order to better comprehend a waste flow of diabetic debris, I use thematic coding to analyze interviews and user generated content in online public forums. I aim to explore creative solutions for individuals with diabetes and others who manage waste by examining potential gaps between policy and practice in everyday life.

\end{abstract}


\newpage
\section{Synopsis for Ashgate Book}




\newpage
\section{Introduction}
\doublespacing
What do indidviduals who generate medical waste do with their waste? Although there are many policies in place at national, state, provincial, and municipal levels, what does the actual practice of personal medical waste management look like on an everyday basis? Diabetes, which is already quite prevalent and growing worldwide, will serve as a case study for understanding attitudes about personal biomedical waste management and peronsal practices of disposal. These practices and attitudes are very much a problem for geography becasuse where one lives or travels affects not only what type of waste one generates have, but also how and where one divests it. This research is primarily situated in North America, making comaprisons between the US and Canada. Examining waste practices on this scale will aid in understanding what happens to this type of waste once it leaves the consumer/user. Medical waste management on a much larger scale, such as hospitals and retirement homes is more often researched and better understood from a urban planning perspective. I will conduct a qualitative analysis of personal medical waste management through interviews, and online communities dedicated to people with diabetes. For this study, I draw on \citet{barr_household_2002} to draw attention to the importance of social attitudes surrounding household waste.
 %How do we create awareness for waste management so that those creating, living with and handling the waste are putting themselves at the least possible risk? 

Diabetes requires constant management and produces copious amounts of biomedical waste. This waste not only requires management in clinics and hospitals where people with diabetes are treated, but also in various private and public spaces, such as homes, public spaces and touristic destinations. The line of treatment an individual diabetic follows determines the quantity and quality of their waste. One fairly universal tool or technology in diabetes management requires the use of sharp lancets and blood test strips, often numerous times a day, to monitor blood glucose levels. Additionally, people with diabetes can produce large amounts of waste including, glass insulin vials, syringes, infusion needles, infusion sets, insulin pens, pods, patches, reservoirs, continuous glucose monitors, wireless transmitters etc… 

This research seeks to understand the retention, disposal and ridding of medical waste on an idividual level for those who regularly generate waste that contains blood, fluids and sharps.
Safe sharps disposal is a concern for not only health and waste management workers, but also important at home, at work, at school, traveling, or in other public places such as hotels, parks, and restaurants. 

While the US Food and Drug Administration discourages people who manage health problems at home from placing loose needles and other sharps (ones not placed in a designated sharps container) in the household or public trash cans or recycling bins, and flushing sharps down the toilet because it puts trash and sewage workers, janitors, housekeepers, and household members at risk of being harmed, the FDA doesn't discourage sharps from being landfilled. Quite the opposite, the FDA encourages a flow of formal and informal sharps containers into the landfill. 

\citet{bouhanick_what_2000} conducted an international study using a questionnaire on what people with diabetes do with their sharps and monitoring waste. They found that \begin{quote} At home: needles, syringes, lancets and reagent strips were thrown directly into the bin in 46.9\%, 49.9\%, 52.2\% and 67.6\% of cases, respectively; and in a closed plastic bottle in 29. 6\%, 28.5\%, 28.9\% and 19.9\% of cases, respectively. Specific containers were used in 8.6\% and 6.3\% of cases for needles and syringes, respectively. 62\% of the bottles and containers were thrown directly into the bin, whereas 15.5\% were returned to a pharmacy (4.5\% taken to hospitals, 2.9\% were burned). At work: 63\% of the patients brought their needles and syringes home for disposal, 6.9\% kept suitable containers at work and 30\% threw their materials directly into local bins (p.851) \end{quote}. Their study shows a clear need for a better way to dispose of these types of household biomedical waste and even more so as the number of people using self injectibles continues to rise.  Additionally, \citet{crawshaw_disposal_2002} and \citet{dallel_disposal_2005} conducted studies  on the disposal of diabetic debris and both concluded that a of lack of user knowledge is largely to blame and there is a serious need for the standardization of sharps disposal and the dissemination of best practices to individuals. \citet{govender_sharps_2012} studied sharps disposal practices in South Africa and likewise highlighted the need for better education of how to dispose of sharps. The underlying theme calls for better education, but even those educated on how to dispose of their daily biomedical waste show low compliance. This in essence comes down to convenience, something not to be take for granted when living with chronic illness. 

Making everyday life more convenient for people who manage chronic illnesses is difficult to weigh against the effect of waste produced from more accurate and more convenient devices \citep{gilg_green_2005}. According to \citet{krisiunas_waste_2011} \begin{quote}
The development of the single-use disposable syringe, initially by Murdoch and subsequently by Becton Dickinson and Monoject, was a major milestone for diabetes care as well as health care and injection safety. What we have now learned in the ensuing years is that all these single-use devices and the various next generation products, including diabetes care products, have spawned an interesting contribution to the global waste stream ( p. 851). \end{quote}

In addition to looking at already published work on household waste \citep{chappells_dustbin_1999} and medical waste and sharps disposal by people with diabetes \citep{bouhanick_what_2000,mcconville_syringe_2002}, I look to public internet forums to answer the question of what people with diabetes actually do with their diabetes related waste. I've found that the most common methods of disposal to be 1) Toss in in the waste bin/garbage (lanfill stream) 2) Participation in sharps collection or drop off program 3) Flushing down toilets and 4) Retaining them for long periods of time. 

Next, I follow up on the flow of waste based on methods of disposal suggested in forums. I move the study from the internet to Kingston, Ontario for a grounded perspective. This is research in progress, so I will focus on waste that flows from a consumer to the Ravensview waste water treatment facility or storm drains in Kingston, ON. I will briefly mention the the other flows, which I am still researching.


While Canada has national standards for sharps disposal and many sharps collection and return services for consumers, it varies by province \citep{walkinshaw_medical_2011}. Canada and Ontario actively divert sharps and medical waste from landfills. Ontario has recently  It aims to be self-sufficient in its management of healthcare waste by relying on regional incineration facilities \citep{anyinam_managing_1994}. In the United States there is no national standard for individuals to dispose of sharps and policy is determined by the state and municipality. The US also has sharps collection services but the cost is typically incurred by the individual. The following is not policy but suggestions from the FDA for disposing of sharps and  US: \begin{quote}Place needles, syringes, lancets, and other sharp objects into hard plastic or metal containers with a screw-on top or other tightly securable lid (e.g., an empty paint can or liquid-detergent container).  Before discarding, reinforce the top with heavy-duty tape. Do not put sharp objects in any container you plan to recycle. Do not use clear plastic or glass containers. Containers should be no more than full. Check with your local waste collection service to make sure these disposal procedures are acceptable in your county. All sharps should be disposed of in rigid puncture-resistant containers such as liquid detergent bottles, bleach bottles or metal containers \citep{fda_center_2014}.\end{quote} 
In the US one can either dispose of sharps and personal medical waste by placing it in a container with duct tape and label for transport to a landfill or one can pay a private service to dispose of the sharps and medical waste. There options are not ideal for the individual nor the wellbeing of a community. Because there is little incentive to dispose of these types of waste in compliance with the local or national standards, a shocking number simply do not.

There is healthy debate in public health and medicine as to sharps disposal. Where healthcare workers were once required to recap needle devices, they are now discourage from doing so. This is no longer considered a best practice because the majority of healthcare workers needle sticks occured during the recapping of the needle \citep{ancona_insulin_1994}. There are concerns for environmental impacts for lanfilling and incineration. Other creative ways to deal with sharps and biomedical waste comes from prodcut inventors such as that of \citet{anyinam_managing_1994}, which is a method and apparatus for sterilization and separation of plastic and non-plastic medical wastes. 

\subsection{Ravensview Waste Water Treatment Facility}
What happens in Kingston Ontario to a needle once it is flushed?
At Ravensview syringes and other diabetic supplies that are flushed undergoes a process that separates plastics from organice materials similar to that of \citet{anthony_method_1994}. 


% `The key to keeping disposal methods cost-effective and reducing emissions is accurate sorting of materials to minimize the waste that needs to be either treated or incinerated, as opposed to being shipped to landfills'', Rubinstein says.
% 
% Canadian regulations should be modified so that all items saturated with blood aren’t automatically considered hazardous waste, Rubinstein adds. “There’s very little evidence or no evidence that it’s actually an infections hazard. A lot of it is the so called ‘ick’ factor. It looks gross. No one wants to see these things ending up in the landfill so the legislation is to have it treated.”
% 
% Rubinstein adds that domestic, residential garbage often contains blood-soaked materials that don’t require special treatment. “Nobody really notices it but when it comes out of the hospital, there’s extra attention.”

Incineration may be required for certain hazardous materials, like pathological waste, but a large portion of personal medical waste can be treated through non-incineration technologies utilizing steam or heat to sterilize the material. 

%{Disposal Practice}

\section{Conclusion}
Who takes on the responsibility for disposal?  Is it the end-user, is it the device manufacturer, is it the product designer, is it the Federal Government, is it state or local government, the waste industry, is it the healthcare provider, or is it the pharmacists or point-of-sale? The answers to these questions rely wholly on geographical location, social attitudes and creativity. \citet{krisiunas_waste_2011} makes interesting comparisons between disposable coffee cups and 
Effects of sharps and biomedical waste management has consequences for people and places. Flows of waste from hospitals and health care providers as well as indiviuals require serious attention from policy makers, product designers, and waste management facilites. 
It seems that education and social attitudes change slowly all the while the waste adds up. Perhaps hydrodegradable casings for the needles and lancets, so the are ready to flush. Is education the answer?
%\section{Flushing}


%\section{Some \LaTeX{} Examples}
%\label{sec:examples}

%\subsection{Sections}

%Use \texttt{section}s and \texttt{subsection}s to organize your document. \LaTeX{} handles all the formatting and numbering automatically. Use \texttt{ref} and \texttt{label} for cross-references --- this is Section~\ref{sec:examples}, for example.

%\subsection{Tables and Figures}

%Use \texttt{tabular} for basic tables --- see Table~\ref{tab:widgets}, for example. You can upload a figure (JPEG, PNG or PDF) using the files menu. To include it in your document, use the \texttt{includegraphics} command (see the comment below in the source code).

% Commands to include a figure:
%\begin{figure}
%\includegraphics[width=\textwidth]{your-figure's-file-name}
%\caption{\label{fig:your-figure}Caption goes here.}
%\end{figure}

%\begin{table}
%\centering
%\begin{tabular}{l|r}
%Item & Quantity \\\hline
%Widgets & 42 \\
%Gadgets & 13
%\end{tabular}
%\caption{\label{tab:widgets}An example table.}
%\end{table}

%\subsection{Quotes}

% From: http://www.ncbi.nlm.nih.gov/pmc/articles/PMC3255123/
% 
% "As disturbing as the litany of unsound world practices in the disposal of medical waste was United Nations Special Rapporteur Calin Georgescu’s conclusion that “only a limited number of countries has developed, or is in the process of developing, a national regulatory framework” to handle the mountain of medical waste now being produced by the world’s health facilities"
% 
% "For the most part, Canada’s hospitals appear to moving away from on-site incinerators toward centralized provincial facilities for the actual sterilization of biomedical waste."



%\subsection{Lists}

\newpage
\singlespacing
\bibliographystyle{apa}
\bibliography{dd}

\end{document}